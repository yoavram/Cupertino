\documentclass[9pt, a4paper, twocolumn]{extarticle}
\usepackage[margin=1in]{geometry} % See geometry.pdf to learn the layout options.
\geometry{a4paper}
%\usepackage[parfill]{parskip} % Activate to begin paragraphs with an empty line rather than an indent
\usepackage{graphicx} % Use pdf, png, jpg, or eps§ with pdflatex; use eps in DVI mode
% TeX will automatically convert eps --> pdf in pdflatex		

\usepackage{amssymb,amsmath,amsthm}
\usepackage{commath}
\usepackage{longtable}
\usepackage[hyphens]{url}
\PassOptionsToPackage{hyphens}{url} % url is loaded by hyperref
\usepackage[unicode=true]{hyperref}
%\usepackage{nameref} % included in hyperref
\usepackage[auth-sc]{authblk}

% messes up nameref
%\usepackage{titlesec}
%\titleformat{\subsection}[runin]
%  {\bf}{\thesection}{1em}{}
  
%SetFonts
% newtxtext+newtxmath
\usepackage{newtxtext} %loads helv for ss, txtt for tt
\usepackage{amsmath}
\usepackage[bigdelims]{newtxmath}
\usepackage[T1]{fontenc}
\usepackage{textcomp}
%SetFonts

% Yoav & Lee commands
\newcommand*{\tr}{^\intercal}
\let\vec\mathbf
\newcommand{\matrx}[1]{{\left[ \stackrel{}{#1}\right]}}
\newcommand{\diag}[1]{\mbox{diag}\matrx{#1}}
\newcommand{\goesto}{\rightarrow}
\newcommand{\dspfrac}[2]{\frac{\displaystyle #1}{\displaystyle #2} }
\newtheorem{theorem}{Theorem}
\newtheorem{corollary}{Corollary}
\newtheorem{lemma}{Lemma}
\newtheorem{remark}{Remark}
\newtheorem{result}{Result}
\renewcommand\qedsymbol{} % no square at end of proof
\newcommand{\cl}{\mathbf{L}}
\newcommand{\cj}{\mathbf{J}}
\newcommand{\ci}{I}


% NatBib
\usepackage[round,colon,authoryear]{natbib}

% Title page
\title{Generation of Variation and Mean Fitness Increase: \\ Necessity is the Mother of Genetic Invention}
%\title{Conditional Generation of Variation and Mean Fitness Increase: \\ Necessity is the Mother of Genetic Invention}
% \title{A General Principle for the Evolution of Fitness-Associated Generation of Variation}

\author[a]{Yoav Ram}
\author[b]{Lee Altenberg}
\author[c]{Uri Liberman}
\author[a]{Marcus W. Feldman}
\affil[a]{Department of Biology, Stanford University, Stanford, CA}
\affil[b]{Information and Computer Sciences, University of Hawai`i at M{\=a}noa, Honolulu, HI}
\affil[c]{School of Mathematical Sciences, Tel Aviv University, Israel}

\date{\today}

% Document
\begin{document}
\maketitle

% Abstract
\begin{abstract}
Generation of variation may be detrimental in well-adapted populations evolving under constant selection.
In a constant environment, genetic modifiers that reduce the rate at which variation is generated by processes such as mutation and migration, succeed.
However, departures from this \emph{reduction principle} have been demonstrated.
Here we analyze a general model of evolution under constant selection where the rate at which variation is generated depends on the individual.
We find that if a modifier allele increases the rate at which individuals of below-average fitness generate variation, then it will increase in frequency and increase the  population mean fitness. This principle applies to phenomena such as stress-induced mutagenesis and condition-dependent dispersal, and exemplifies \emph{``Necessity is the mother of genetic invention.''}
\end{abstract}

% Introduction
\section*{Introduction}

According to the \emph{reduction principle}, in populations at a balance between natural selection and a process that generates variation (i.e. mutation, migration, or recombination), selection favors neutral modifiers that decrease the rate at which variation is generated. 
The \emph{reduction principle} was demonstrated for modifiers of the recombination~\citep{Feldman1972}, mutation~\citep{Liberman1986a}, and migration~\citep{Feldman1986}.
These results were unified in a series of studies ~\citep{Altenberg1984,Altenberg1987,Altenberg2009,Altenberg2012b,Altenberg2012a,Altenberg2017}.

The latter studies have established the conditions for a \emph{unified reduction principle} by neutral genetic modifiers:
(i) effectively infinite population size, (ii) constant-viability selection, (iii) a population at an equilibrium, and (iv) \emph{linear variation} -- the equal scaling of transition probabilities by the modifier.
A departure from the latter assumption occurs if two variation-producing processes interact~\citep{Feldman1980,Altenberg2012b}.
Departures from the \emph{reduction principle} have also been demonstrated when conditions (i)-(iii) are not met, see for example~\citet{Holsinger1986} and references therein.

Another departure from the \emph{linear variation} assumption of the \emph{reduction principle} for mutation rates involves a mechanism by which the mutation rate increases in individuals of low fitness -- a mechanism first observed in stressed bacteria~\citep{Foster2007}, although not in a constant environment.
\citet{Ram2012} demonstrated that even in a constant environment, increasing the mutation rate of individuals with below-average fitness increases the population mean fitness, rather than decreases it.
Their analysis assumed  infinite population size and fitness  determined by the number of mutant alleles accumulated in the genotype.
In their models, the only departure from the \emph{reduction principle} assumptions was the unequal scaling of mutation probabilities between different genotypes introduced by the correlation between the mutation rate and fitness.
A similar result has been demonstrated for conditional dispersal~\citep[Th.~39]{Altenberg2012b}, fitness-associated recombination~\citep{Hadany2003a} and for condition-dependent sexual reproduction~\citep{Hadany2007a}, and evidence suggests that both mechanisms are common in nature~\citep{Ram2016}.

\citet{Ram2012} stated that their result represents a departure from the \emph{reduction principle}, but did not explain this departure.
Their analysis was specific to a model that classified individuals by the number of mutant alleles in their genotype, similar to models studied by~\citet{Kimura1966a} and~\citet{Haigh1978}.
Moreover, their argument was based on the expected increase of the stable population mean fitness, rather than on the invasion success of modifier alleles that modify the mutation rate~\citep[i.e., analysis of \emph{evolutionary genetic stability}, see][]{Eshel1982, Lessard1990}.

Here, we present an evolutionary model where the the type of the individual determines both its fitness and the rate at which it generates variation.
Our results show that the population mean fitness increases if individuals with below-average fitness produce more variation than individuals with above-average fitness,
and that modifier alleles that induce below-average individuals to produce more variation are favored by natural selection.

% Model
\section*{Models}\label{sec:models}

\paragraph*{General model.}\label{sec:general_model}
Consider a large  population with an arbitrary set of types $A_1, A_2, \ldots, A_n$.
The frequency and fitness of individuals of type $A_k$ are $f_k$ and $w_k$, respectively.
The probability that an individual of type $A_k$ will transition to some other type is $C_k$, and given a transition occurs, the probability that it will transition to type $A_j$ is $M_{j,k}$.
Therefore, the change in the frequencies of type $A_k$ is described by the transformation $f \to f'$: 
\begin{equation}
\bar{w} f'_k = (1-C_k) w_k f_k + \sum_{j=1}^{n}{C_j M_{k,j} w_j f_j},
\label{eq:model_sum}
\end{equation}
or in matrix form
\begin{equation}
\bar{w} f' = (\ci - C + MC)D f,
\label{eq:model}
\end{equation}
where $f=(f_1, f_2, \ldots, f_n)$ is a frequency vector with $f_k\ge 0$ and  $\sum_{k=1}^n{f_k} = 1$;
$D$ is a positive diagonal matrix with  elements $w_k$ such that $w_k \ne w_j$ for some $k \ne j$; 
%and we can assume $0 < w_k \le 1$, i.e., fitness values are positive and relative to the fitness of the fittest type; 
$C$ is a positive diagonal  matrix with  elements $C_k$;
$M$ is a primitive column-stochastic matrix: $M_{j,k} \ge 0$ for all $j,k$, $\sum_{j=1}^n {M_{j,k}} = 1$ for all $k$, and $(M^n)_{j,k} > 0$ for all $j,k$ for some natural $n$;
$\ci$ is the $n\times n$ identity matrix;
and $\bar{w}$ is the normalizing factor such that $\sum_{k=1}^n{f'_k}=1$ and is equal to the population mean fitness $\bar w=\sum_{k=1}^n f_kw_k$.

The types $A_k$ can represent a single or multiple haploid genetic loci or non-genetic traits.
Importantly, type transmission is vertical and uni-parental (the type is transmitted from a single parent to the offspring) and is independent of the frequencies of the other types.
Specifically, this model precludes processes such as recombination, social learning, sexual outcrossing, and horizontal or oblique transmission, as these processes are frequency-dependent~\cite[pg.~54]{Cavalli-Sforza1981}.

Transition between types is determined by a combination of two effects:
(i) the probability of transitioning \emph{out} of type $A_k$ is determined by $C_k$;
(ii) given a transition out of type $A_k$, the distribution of the destination types $A_i$ is given by $M_{i,k}$ (note the index order).
Importantly, different types can have different rates.
That is, $C_i \ne C_j$ for some $i,j$. The case $C_i = C_j$ for all $i,j$ is covered by the \emph{reduction principle}~\citep[see][]{Altenberg2017}.

In the following section we present four examples of the model (Eq.~\ref{eq:model}) that apply to mutation, migration, and learning.

% Mutation model 1
\paragraph*{Mutation model 1.}\label{sec:mut_model_1}

Here we consider a large population of haploids and a trait determined by a single genetic locus with $n$ possible alleles $A_1, A_2, \ldots, A_n$ and corresponding fitness values $w_1, w_2, \ldots, w_n$.
The mutation rates $C_k$ of individuals with allele $A_k$ are potentially different;
specifically, with probability $1-C_k$, the allele $A_k$ does not mutate, and with probability $\frac{C_k}{n-1}$, the allele $A_k$ mutates to $A_j$ for any $j \ne k$.
This is an extension of a model studied by~\citet{Altenberg2017} that allows for the mutation rate of $A_k$, $C_k$, to depend on properties of the allele $A_k$.

Let the frequency of $A_k$ in the present generation be $f_k$ with $f_k \ge 0$ and $\sum_{k=1}^{n}{f_k}=1$.
Then after selection and mutation, $f'_k$ in the next generation is given by
\begin{equation}
f_k' = \big(1 - C_k\big) \frac{w_k}{\bar{w}} f_k + \frac{1}{n-1} \sum_{j \ne k}{C_j \frac{w_j}{\bar{w}} f_j},
\label{eq:mutation_model_1}
\end{equation}
for all $k=1, 2, \ldots, n$, where $\bar{w} = \sum_{k=1}^{n}{f_k w_k}$ is the population mean fitness.

This model is a special case of the general model (Eq.~\ref{eq:model}) where
\begin{equation} 
{M} = \frac{1}{n-1} \begin{bmatrix}
0 & 1 & 1 & \ldots & 1 \\
1 & 0 & 1 & \ldots & \vdots \\
1 & 1 & 0 & \ldots & \vdots \\
\vdots & \vdots & \vdots & \ddots & \vdots \\
1 & \ldots & \ldots & \ldots & 0
\end{bmatrix},
\end{equation}
with zeros on the diagonal and $\frac{1}{n-1}$ elsewhere.
Note that here ${M}$ is irreducible and primitive.

% Mutation model 2
\paragraph*{Mutation model 2.}\label{sec:mut_model_2}

Again, we consider a large population of haploids, but here individuals with genotype $A_k$ are characterized by the number $k$ of deleterious or mutant alleles in their genotype, where $0 \le k \le n$.
Specifically, the fitness of individuals with $k$ mutant alleles is $w_k$ ($w_0 > w_1 > \ldots > w_n$),
and the probability $C_k$ that a mutation occurs in individuals with $k$ mutant alleles depends on $k$.
When a mutation occurs it is \emph{deleterious} with probability $\delta$, generating a mutant allele and converting the individual from $A_k$ to $A_{k+1}$,
or it is \emph{beneficial} with probability $\beta$, converting the individual from $A_k$ to $A_{k-1}$.
Note that such beneficial mutations can be either compensatory or back-mutations, and that mutations are neutral with probability $1-\delta-\beta$.
We assume that both the deleterious and the beneficial mutation rates are low, such that two mutations are unlikely to occur in the same individual in one generation: $C_k(\delta + \beta) \ll 1$ for all $k=1, \ldots, n$.
This model has been analyzed by~\citet{Ram2012}.

Let the frequency of $A_k$ in the present generation be $f_k$ with $f_k \ge 0$ and $\sum_{k=0}^{n}{f_k}=1$.
Then after selection and mutation $f'_k$ in the next generation is given by
\begin{equation}
\begin{aligned}
f_0' &= \big(1 - \delta C_0\big) \frac{w_0}{\bar{w}} f_0 + \beta C_{1} \frac{w_{1}}{\bar{w}} f_{1}, \\
f_k' &= \big(1 - (\delta+\beta) C_k\big) \frac{w_k}{\bar{w}} f_k + \\
	 & \delta C_{k-1} \frac{w_{k-1}}{\bar{w}} f_{k-1} + 
	 \beta C_{k+1} \frac{w_{k+1}}{\bar{w}} f_{k+1}, \\
f_n' &= \big(1 - \beta C_n\big) \frac{w_n}{\bar{w}} f_n + \delta C_{n-1} \frac{w_{n-1}}{\bar{w}} f_{n-1},	 
\end{aligned}
\label{eq:mutation_model_2}
\end{equation}
for $k=1,2,\ldots, n-1$.
Here $\bar{w}=\sum_{k=0}^{n}{f_k w_k}$ is the population mean fitness.

Therefore, setting 
\begin{equation}
{M} = \begin{bmatrix}
1-\delta & \beta & 0 &  \cdots & 0\\
\delta & 1-\delta-\beta & \beta &  \ddots & 0\\
0 & \delta & 1-\delta-\beta & \ddots & 0 \\
\vdots & \ddots & \ddots & \ddots & \beta \\
0 & 0 & 0 & \delta & 1-\beta
\end{bmatrix},
\end{equation}
with the beneficial, neutral, and deleterious mutation probabilities on the three main diagonals and zeros elsewhere, 
Eq.~\ref{eq:mutation_model_2} can be viewed as a special case of Eq.~\ref{eq:model}.
Here, too, ${M}$ is irreducible and primitive as long as $\delta, \beta > 0$.

% Migration model
\paragraph*{Migration model.}\label{sec:mig_model}

In this case we consider a large population of haploids that occupy $n$ demes, $A_1, \ldots, A_n$.
Let the frequencies of individuals in deme $A_k$ be $f_k$ with $f_k \ge 0$ and $\sum_{k=1}^{n}{f_k} = 1$.
The fitness of individuals in deme $A_k$ is $w_k$, but the entire population comes together for reproduction, and therefore reproductive success is determined by competition among individuals of all demes -- this has been termed \emph{hard selection}~\citep{Wallace1975,Karlin1982}.

After reproduction, offspring of individuals from deme $A_k$ return to their parental deme with probability $1-C_k$, or migrate to a different deme $A_j$ with probability $C_k M_{j,k}$, where the matrix $M$ is primitive, $C_k > 0$, $M_{j,k} \ge 0$, and $\sum_{j=1}^{n}{M_{j,k}} = 1$ for all $k=1, \ldots, n$.
Therefore, $1-C_k$ can be interpreted as a \emph{homing rate}.

Following selection and migration the new frequencies $f'_k$ are given exactly by Eq.~\ref{eq:model_sum}.
If the columns of ${M}$ are identical
\begin{equation}
{M} = \begin{bmatrix}
m_1 & m_1 & \ldots & m_1 \\
m_2 & m_2 & \ldots & m_2 \\
\vdots & \vdots & \vdots & \vdots \\
m_n & m_n & \ldots & m_n
\end{bmatrix},
\end{equation}
with $m_k>0$ and $\sum_{k=1}^{n}{m_k}=1$,
then $m_k$ can be considered the relative population size of deme $A_k$ --
this is the non-homogeneous extension of Deakin's \emph{homing model}~\citep{Deakin1966,Karlin1982}.

Similarly, if demes are arranged in a circle, for example around a lake, 
then we can denote the probability $p_k$ of migrating $k$ demes away from the parental deme (conditioned on migration which occurs with probability $C_k$)
and ${M}$ takes the form
\begin{equation}
{M} = \begin{bmatrix}
p_0 & p_1 & \ldots & p_{n-1} \\
p_{n-1} & p_0 & \ldots & p_{n-2} \\
\vdots & \vdots & \ddots & \vdots \\
p_1 & p_2 & \ldots & p_0
\end{bmatrix},
\end{equation}
where $p_k > 0$ and $\sum_{k=0}^{n-1}{p_k}=1$.

% Learning model
\paragraph*{Learning model.}\label{sec:learn_model}

In our final example, we consider a large population and an integer phenotype $k$ where $1 \le k \le n$.
Individuals are characterized by their initial and mature phenotypes~\citep[pg.~94]{Boyd1985}.
Fitness is determined by the mature phenotype: the fitness of an individual with mature phenotype $k$ is $w_k$.

An offspring's initial phenotype is acquired by learning the mature phenotype of its parent (assuming uni-parental transmission).
In individuals with initial phenotype $k$, the mature phenotype is the same as the initial phenotype with probability $1-C_k$, and is modified by individual learning or \emph{exploration}~\citep{Borenstein2008} with probability $C_k$.
Such individual exploratory learning, which can be considered either intentional or the result of incorrect learning, modifies initial phenotype $k$ to mature phenotype $j$ with probability $M_{j,k}$.

Therefore, if the frequency of individuals with mature phenotype $k$ in the current generation is $f_k$, then the frequency in the next generation $f'_k$ is
\begin{equation}
f'_k = (1-C_k) f_k \frac{w_{k}}{\bar{w}} + \sum_{j=1}^{n}{C_j M_{k,j} \frac{w_{j}}{\bar{w}} f_j },
\end{equation}
where $\bar{w} = \sum_{k=1}^{n}{f_k w_{k}}$ is the population mean fitness.

For example, in the case of \emph{symmetric individual learning}~\citep{Borenstein2008}, learning is parameterized by its breadth of exploration $b$ and the mature phenotype $j$ is randomly drawn from $2b+1$ phenotypes symmetrically and uniformly distributed around the initial phenotype $k$, with the limitation that any "spillover" of phenotypes below $1$ or above $n$ is "absorbed" by those boundaries.
This "absorption" ensures $M$ is column-stochastic. 
In other words, given initial phenotype $k$, the probability for maturation to phenotype $j$ such that $k-b \le j \le k+b$ is $1/(2b+1)$, but any phenotype $j<1$ actually becomes $j=1$ and any phenotype $j>n$ actually becomes $j=n$.
The probability for maturation to other phenotypes is $0$.

For instance, with $n=5$ and $b=1$ we have
\begin{equation}
M = \begin{bmatrix}
2/3 & 1/3 & 0 & 0 & 0 \\
1/3 & 1/3 & 1/3 & 0 & 0 \\
0 & 1/3 & 1/3 & 1/3 & 0 \\
0 & 0 & 1/3 & 1/3 & 1/3 \\
0 & 0 & 0 & 1/3 & 2/3
\end{bmatrix},
\end{equation}
and with $n=6$ and $b=2$ we have
\begin{equation}
M = \begin{bmatrix}
3/5 & 2/5 & 1/5 & 0 & 0 & 0 \\
1/5 & 1/5 & 1/5 & 1/5 & 0 & 0 \\
1/5 & 1/5 & 1/5 & 1/5 & 1/5 & 0 \\
0 & 1/5 & 1/5 & 1/5 & 1/5 & 1/5 \\
0 & 0 & 1/5 & 1/5 & 1/5 & 1/5 \\
0 & 0 & 0 & 1/5 & 2/5 & 3/5
\end{bmatrix}.
\end{equation}

% Results
\section*{Results}

% Mean fitness principle
\paragraph*{Mean fitness principle.}

We first focus on the stable population mean fitness.
We will show that if the transition rate from types with below-average fitness increases, then the stable population mean fitness increases, too.

Write the equilibrium  frequency vector $f$ in Eq.~\ref{eq:model} as $\hat v$ and the stable population mean fitness as $\hat{\bar w}$, then
\begin{equation}\label{eq:model_equilibrium}
\hat{\bar w} \hat v = (I-C+MC)D \hat v.
\end{equation}
Note that (i) the existence and uniqueness of $\hat{\bar w}$ and $\hat v$ are guaranteed by the \emph{Perron-Frobenius theorem}~(\cite{Otto2007}) because $(I-C+MC)D$ is a non-negative primitive matrix; 
(ii) the global stability of this equilibrium is proven in~\nameref{sec:AppC}.

The following result constitutes a \emph{mean fitness principle} for the
sensitivity of the equilibrium mean fitness $\hat{\bar w}$ to changes
in $C_k$, the probability of transition from $A_k$.
\medskip

\begin{result}[Mean fitness principle]\label{result:mfp}
Let $\hat{\bar w}$ be the leading eigenvalue of $(I-C+MC)D$,
and $\hat u$ and $\hat v$ be the corresponding positive left and right eigenvectors, such that $\sum_{k=1}^{n}\hat v_k=1$ and $\sum_{k=1}^{n}\hat u_k\hat v_k=1$.
Then,
\begin{equation}\label{eq:mfp}
\frac{\partial \hat{\bar w}}{\partial C_k} = 
\frac{\hat u_k \hat v_k}{C_k} (\hat{\bar w} - w_k),
\end{equation}
or in simpler terms,
\begin{equation}\label{eq:mfp_sign}
sign\frac{\partial \hat{\bar w}}{\partial C_k} = 
sign(\hat{\bar w} - w_k).
\end{equation}

Therefore increased transition from type $k$ will increase the stable population mean fitness if the fitness of type $k$ is below the stable population mean fitness.
\end{result}

\begin{proof}
Using the formula in~\citet{Caswell1978} (see eq.~\ref{eq:Caswells_formula} in~\nameref{sec:AppA}),
\begin{equation}
\frac{\partial \hat{\bar w}}{\partial C_k} = 
\hat u\tr \frac{\partial (I-C+MC)D}{\partial C_k} \hat v.
\end{equation}

Let $e_k$ and $e\tr_k$ be the column and row vectors with 1 at position $k$ and 0 elsewhere, $Z_k = e_k e\tr_k$ be the matrix with 1 at position $(k,k)$ and 0 elsewhere, and $[M]_k$ be the $k$-th column of $M$.

Then,
\begin{equation}\label{eq:mfp_intermediate}
\begin{aligned}
\hat u\tr \frac{\partial (I-C+MC)D}{\partial C_k} \hat v = 
\hat u\tr (0 - Z_k + M Z_k)D \hat v = \\
-\hat v_k \hat u_k w_k + \hat v_k w_k \hat u\tr[M]_k = \\
\hat v_k w_k (\hat u\tr[M]_k - \hat u_k). 
\end{aligned}
\end{equation}

The corresponding equation to Eq.~\ref{eq:model_equilibrium} for the left
eigenvector $u$ is 
\begin{equation}
\hat u\tr \hat{\bar w} = \hat u\tr (I - C + MC) D,
\end{equation}
which gives us a relation between $\hat{\bar w}$ and the $k$
element of $\hat u$:
\begin{equation}
\hat u_k \hat{\bar w} = (1-C_k) w_k \hat u_k + C_k w_k \hat u\tr [M]_k.
\end{equation}
Multiplying both sides by $\hat v_k$  and rearranging, we get 
\begin{equation}
\frac{\hat u_k \hat v_k}{C_k} (\hat{\bar w} - w_k) = \hat v_k w_k (\hat u\tr [M]_k - \hat u_k),
\end{equation}
which when substituted into Eq.~\ref{eq:mfp_intermediate} yields:
\begin{equation}
\frac{\partial \hat{\bar w}}{\partial C_k} = 
\frac{\hat u_k \hat v_k}{C_k} (\hat{\bar w} - w_k).
\end{equation}
Finally, since $\hat u_k, \hat v_k, C_k > 0$, we have
\begin{equation}
sign \frac{\partial \hat{\bar w}}{\partial C_k} = 
sign (\hat{\bar w} - w_k),
\end{equation}
which completes the proof.
\end{proof}

The above result provides a condition for the effect of changing $C_k$, the probability for transition from $A_k$, on the stable population mean fitness.
Specifically, if $A_k$ individuals have below-average fitness, then increasing $C_k$ will increase the population mean fitness.

% Multiple types
\paragraph*{}
We turn our attention to the case where the transition rates from a subset $K$ of the types are correlated, that is, $C_j=C_i$ for $i,j \in K$.
In this case, Eq.~\ref{eq:mfp} leads directly to the following.

\begin{corollary}
The sensitivity of the stable population mean fitness to change in the rate of transition $\tau$ from types in $K$ is
\begin{equation}\label{eq:mfp_K}
\begin{aligned}
\frac{\partial \hat{\bar w}}{\partial \tau} = 
\hat u\tr \Big( \sum_{k \in K}{\frac{\partial (I - C + MC)D}{\partial \tau}} \Big) \hat v = \\
\frac{1}{\tau} \sum_{k \in K}{\hat u_k \hat v_k (\hat{\bar w} - w_k)},
\end{aligned}
\end{equation}
and
\begin{multline}\label{eq:mfp_sign_threshold}
sign \frac{\partial \hat{\bar w}}{\partial \tau} = 
sign \sum_{k \in K}{\hat u_k \hat v_k (\hat{\bar w} - w_k)} = \\ 
sign \Big(\hat{\bar w} - \frac{\sum_{k \in K}{\hat u_k \hat v_k w_k}}{\sum_{k \in K}{\hat u_k \hat v_k}}\Big).
\end{multline}
\end{corollary}

Therefore, increased transition from types in $K$ will increase the stable population mean fitness if the average fitness of individuals descended from types in $K$ is below the stable population mean fitness.
For example, \citet[Appendix~B]{Ram2012} considered individuals that are grouped by the number of their accumulated mutant alleles, $k$ (see \nameref{sec:mut_model_2}), 
and the effect of increasing the mutation rate in individuals with at least $\pi$ mutant alleles.
According to Eq.~\ref{eq:mfp_sign_threshold}, this will result in increased stable population mean fitness if individuals with $\pi$ or more mutant alleles have below-average fitness. 

% Reproductive value principle
\paragraph*{Reproductive value principle.}

An interesting interpretation of Eq.~\ref{eq:mfp_intermediate} is
\begin{equation}
\frac{\partial \hat{\bar w}}{\partial C_k} = 
\hat v_k w_k (\hat u\tr[M]_k - \hat u_k) = 
w_k \Big(\sum_{j=1}^{n}{\hat u_j M_{j,k} \hat v_k} - \hat u_k \hat v_k\Big).
\end{equation}
Here, $\hat u_k$ can be regarded as the \emph{reproductive value} of type $k$~\citep[pg.~27]{Fisher1930}, which gives the relative contribution of type $k$ to the long-term population (see~\nameref{sec:AppB}).
Consequently, $\hat u_k \hat v_k$ is the \emph{ancestor frequency} of type $k$~\citep{Hermisson2002}, namely the fraction of the equilibrium population descended from type $k$.
The sum $\sum_{j=1}^{n}{\hat u_j M_{j,k} \hat v_k}$ can be similarly interpreted as the fraction of the equilibrium population descended from individuals that transitioned from type $k$ to another type (via the $k$ column of the transition matrix $M$), conditioned on transition occurring.

Since $w_k>0$, from Eq.~\ref{eq:mfp_intermediate} we have the following corollary.

\begin{corollary}[Reproductive value principle]\label{cor:rvp}
In the notation of Result~\ref{result:mfp},
\begin{equation}
sign \frac{\partial \hat{\bar w}}{\partial C_k} = 
sign (\hat u\tr [M]_k - \hat u_k),
\end{equation}
where $[M]_k$ is the $k$-th column of $M$.

Therefore, increased transition from type $k$ will increase the stable population mean fitness if the fraction of the population descended from type $k$ is expected to increase due to a transition to another type. 
\end{corollary}

Corollary~\ref{cor:rvp} sheds light on why we require $M$ to be primitive.
If $M$ is primitive then individuals of type $k$ can transition into any other type in a finite number of generations.
So individuals with below-average fitness can have descendants with above-average fitness, and increased generation of variation in these individuals will increase the stable population mean fitness.
In contrast, if $M$ is imprimitive, individuals with below-average fitness are "doomed" and increasing the generation of variation in these individuals can only hasten their removal from the population.
For example, if we set $\beta=0$ in \nameref{sec:mut_model_2}, $M$ becomes triangular and imprimitive, and the stable mean fitness becomes $\bar{w} = (1-\delta C_0)w_0$, which is not affected by changes in $C_k$ for $k \ge 1$~\citep[see also][Fig.~1A]{Agrawal2002,Ram2012}.

% Evolutionary stability
\paragraph*{Evolutionary genetic stability.}

% Modifier model
We now focus on a neutral modifier locus completely linked to the types $A_k$, with no direct effect on fitness, and whose sole function is to determine $C_k$, the rates of transition from the different types.
We will show that modifier alleles that increase the stable population mean fitness in accordance with Result~\ref{result:mfp} are favored by natural selection.

\paragraph*{Modifier model.}
Consider the case of two modifier alleles, $m$ and $M$, inducing different transition probabilities $C=\diag{C_1, \ldots, C_n}$ and $\tilde{C}=\diag{\tilde{C}_1, \ldots, \tilde{C}_n}$, respectively.
The frequencies of type $A_k$ linked to modifier $m$ or $M$ are $f_k$ and $g_k$, respectively, where $\sum_{k=1}^{n}{(f_k + g_k)}=1$.
$\bar{w}$ now ensures that $\sum_{k=1}^{n}{(f_k' + g_k')}=1$, 
and the rest of the model parameters are the same as in Eq.~\ref{eq:model}.

The frequencies in the next generation, $f'$ for allele $m$ and $g'$ for allele $M$, are given by
\begin{equation}
\begin{cases}
\bar{w} f' &= (I-C+MC)D f \\
\bar{w} g' &= (I-\tilde{C}+M\tilde{C})D g
\end{cases}.
\label{eq:modifier_model}
\end{equation}
Here, $\bar{w}=\sum_{k=1}^{n}{(f_k + g_k) w_k}$ is the mean fitness of the entire population. 
Note that Eq.~\ref{eq:model} is the special case of Eq.~\ref{eq:modifier_model} where allele $M$ is absent, i.e. $g_k=0$ for all $k$.

We have found a condition for a modifier allele that controls that transition rate $C_k$ from type $A_k$ to increase the stable population mean fitness (Result~\ref{result:mfp}).
Could such an allele increase in frequency when initially rare in the population?
To answer this we analyze the stability of resident modifier allele $m$ with transition rates $C_k$ to the invasion by a modifier allele $M$ with rates $\tilde{C}_k$, see Eqs.~\ref{eq:modifier_model}.

The equilibrium of Eqs.~\ref{eq:modifier_model} when modifier allele $M$ is absent from the population is $(\hat v,0)$, where $\hat v$ is given in Eq.~\ref{eq:model_equilibrium} and $g_k=0$ for all $k$. 
The stability of allele $m$ to invasion by allele $M$ is determined by the leading eigenvalue $\lambda_1$ of $\cl_{ex}$ the external stability matrix of the equilibrium $(\hat v,0)$, which, in turn, is determined by the Jacobian $\cj$ of the system in Eqs.~\ref{eq:modifier_model} evaluated at the equilibrium $(\hat v,0)$, where
\begin{equation}
\cj = \begin{pmatrix}\cl_{in} & 0 \\ 0 & \cl_{ex} \end{pmatrix},
\end{equation}
and $\cl_{in}$ is the local stability matrix of the equilibrium $(\hat v,0)$ in the space $\sum_{k=1}^n{v_k}=1$.
The zero block matrices are due to the complete linkage between the modifier and the types $A_k$ and to the lack of transition (i.e., mutation) between the modifier alleles.

$\cl_{ex}$ can be written as
\begin{equation}\label{eq:Lex}
\cl_{ex} = (I - \tilde{C} + M \tilde{C}) D /\hat{\bar w},
\end{equation}
where $\hat{\bar w}$ is the stable population mean fitness in the absence of modifier allele $M$.
Note that, in general, $\hat{\bar w}$ is not an eigenvalue of $\cl_{ex}$.
\medskip

\begin{result}[Evolution of increased genetic variation]
Let $\lambda_1$ be the the leading eigenvalue of the external stability matrix $\cl_{ex}$.
If the transition rates induced by the modifier alleles $m$ and $M$ are equal, i.e., $\tilde{C}_k=C_k$ for all $k$, then 
\begin{equation}
\lambda_1=1,
\end{equation}
and
\begin{equation}
sign \frac{\partial \lambda_1}{\partial \tilde{C}_k}\bigg\rvert_{\tilde{C}_k = C_k} = 
sign(\hat{\bar w} - w_k).
\end{equation}

Therefore, an initially rare modifier allele $M$ with transition rates slightly different from the resident allele $m$ can successfully invade the population ($\lambda_1>1$) if $M$ increases the probability of transition from types with below-average fitness, thereby increasing the stable population mean fitness.
\end{result}

\begin{proof}
Substituting $\tilde{C}=C$ in Eq.~\ref{eq:Lex} and multiplying both sides by $\hat{\bar w}$,
\begin{equation}
\hat{\bar w} \cl_{ex}\big\rvert_{\tilde{C}_k = C_k} = (I-C+MC) D,
\end{equation}
and since $\hat{\bar w}$ is the leading eigenvalue of the RHS (see Eq.~\ref{eq:model_equilibrium}),
the leading eigenvalue of $\cl_{ex}\big\rvert_{\tilde{C}_k = C_k}$ is $\lambda_1=1$.

Now, applying Result~\ref{result:mfp} (eq.~\ref{eq:mfp_sign}) to Eq.~\ref{eq:Lex}, the sign of the derivative of $\lambda_1$ with respect to $\tilde{C_k}$ is
\begin{equation}
sign \frac{\partial \lambda_1}{\partial \tilde{C}_k} =
sign\Big(\lambda_1 - \frac{w_k}{\hat{\bar w}}\Big).
\end{equation}
Thus
\begin{multline}
sign \frac{\partial \lambda_1}{\partial \tilde{C}_k}\bigg\rvert_{\tilde{C}_k = C_k} = 
sign\Big(1 - \frac{w_k}{\hat{\bar w}}\Big) =  
sign(\hat{\bar w} - w_k),
\end{multline}
since $\hat{\bar w}>0$.
This completes the proof.
\end{proof}

% Reduction principle & mutational loss
\paragraph*{Reduction principle and mutational loss.}

Note that if the modifier has the same effect on all types,
then we can substitute $C = \mu I$ (with $\mu > 0$) in Eq.~\ref{eq:model_equilibrium}, and proceeding as in Eq.~\ref{eq:mfp_K}, we find a relationship previously described by~\citet[eq.~24]{Hermisson2002},
\begin{equation}\label{eq:mutational_loss}
\frac{\partial \hat{\bar{w}}}{\partial \mu} = 
-\frac{1}{\mu}\Big(\sum_{k=1}^{n} \hat u_k \hat v_k w_k - \hat{\bar{w}} \Big) = -\frac{1}{\mu} G,
\end{equation}
where $G$, equal to the difference between the \emph{ancestral mean fitness} $(\sum_{k=1}^{n} \hat u_k \hat v_k w_k)$ and the \emph{stable population mean fitness} $(\hat{\bar{w}})$, is called the \emph{mutational loss}~\citep{Hermisson2002}.

If an invading modifier allele $M$ reduces the transition probability compared to the resident allele $m$, i.e. $C_k = \hat \mu$ and $\tilde{C}_k = \mu$ for all $k$ in Eq.~\ref{eq:Lex} and $\mu < \hat \mu$, then $\cl_{ex}$ becomes
\begin{equation}
\cl_{ex} = ((1 - \mu)I + \mu M) D /\hat{\bar w},
\end{equation} 
and we can we can expect that $M$ will be favored by natural selection.
That is, the \emph{unified reduction principle} is in effect~\cite[eqs. 65, 72]{Altenberg2017}.
Thus, we can conclude that the \emph{mutational loss} $G$ is positive and that the \emph{ancestral mean fitness} is higher then the \emph{stable population mean fitness}.


% Discussion
\section*{Discussion}

We have shown that under constant-viability selection and in an effectively infinite haploid population at mutation-selection or migration-selection equilibrium, the stable population mean fitness increases if individuals with below-average fitness increase the rate at which variation is generated. Furthermore, modifier alleles that  increase generation of variation in such individuals are favored by natural selection.
These results apply as long as there is a chance for the variation-generating process to transform an individual with below-average fitness into one with above-average fitness (e.g. $M$ in Eq.~\ref{eq:model} is primitive).

We have given several examples of variation-generating processes for which this principle applies -- namely mutation, migration, and learning (see \emph{\nameref{sec:models}} section) -- but our model may apply to other processes as well.
For example, the \emph{reduction principle} applies to ecological models of dispersal, and~\citet{Gueijman2013} have demonstrated that even in homogeneous environments, fitness-associated dispersal increases the mean fitness of diploid populations and is favored by selection over uniform dispersal.
Similarly, if the transmission fidelity of culturally-transmitted traits depends on the type or fitness of the transmitting individual, we expect that our results will hold (see \nameref{sec:learn_model}).

Eq.~\ref{eq:mfp} is a generalization of a result of~\citet[Eq.~4]{Ram2012}.
\citeauthor{Ram2012} modeled the accumulation of mutant alleles in a population (see \nameref{sec:mut_model_2}).
Using Eq.~\ref{eq:Caswells_formula} in \nameref{sec:AppA} and a recursion on the ratios of the reproductive values \cite[see][eqs.~A5-6]{Ram2012}, they concluded that at the mutation-selection balance, if individuals with below-average fitness ($w_k < \hat{\bar w}$) increase their mutation rate, then the population mean fitness will increase -- a result generalized by our \emph{\nameref{result:mfp}} in Eqs.~\ref{eq:mfp_sign} and~\ref{eq:mfp_sign_threshold}.

Our analysis focuses on populations at equilibrium.
Nevertheless, it has been demonstrated that during adaptive evolution (i.e., in non-equilibrium populations), a modifier that increases the mutation rate of maladapted individuals can be favored by selection~\citep{Ram2012} and increase the adaptation rate~\citep{Ram2014},
and empirical evidence suggests that \emph{stress-induced mutagenesis} is common in bacteria and yeast, and may be prevalent in plants, flies, and human cancer cells~\citep{Rosenberg2012,Fitzgerald2017b}.
Similar theoretical results have been demonstrated for a modifier that increases the recombination rate in maladapted individuals~\citep{Hadany2003b,Hadany2003a}.

% Conclusions
\paragraph*{Conclusions.}

General examples of departures from the \emph{reduction principle} are rare~\citep{Altenberg2017}.
For example, it has been demonstrated that higher mutation, recombination, and migration rates are favored by selection in populations evolving under fluctuating selection; see~\citet{Carja2014} and references therein.
Here we have provided another general example, which suggests that a modifier allele that causes individuals with below-average fitness to increase the rate at which variation is generated, will be favored by selection and will lead to increased population mean fitness.

\section*{Appendices}

% Appendix: Caswell formula
\subsection*{Appendix A}\label{sec:AppA}

\citet{Caswell1978} gave a \emph{formula for the sensitivity of the population growth rate to changes in life history parameters}.
In this formula, the \emph{population growth rate} is the leading eigenvalue of the population transformation matrix $T$, the \emph{life history parameters} are elements of $T$, and the \emph{sensitivity} is the derivative of the former with respect to the latter.
This is a useful formula~\citep[ch.~10]{Caswell1978,Hermisson2002,Ram2012,Otto2007}, and therefore we reproduce it here.

\begin{lemma}
$T$ be a non-negative matrix with leading eigenvalue $\lambda$ and left and right eigenvectors $\hat u$ and $\hat v$ such that $\sum{\hat v_k}=1$ and $\hat u\tr \hat v = \sum{\hat u_k \hat v_k} = 1$.
Then the sensitivity of $\lambda$ to changes in any element $t$ of the matrix $T$ is
\begin{equation}\label{eq:Caswells_formula}
\frac{\partial \lambda}{\partial t} = 
\hat u\tr \frac{\partial T}{\partial t} \hat v
\end{equation}
\end{lemma}

\begin{proof} 
Using the lemma assumptions,
$\lambda = \lambda \hat u\tr \hat v = \hat u\tr \lambda \hat v = \hat u\tr T \hat v$ and differentiating both sides we get $\partial \lambda = \partial (\hat \hat u\tr T \hat v)$.
Using the product rule (once in each direction),

\begin{multline}
\partial (\hat u\tr T \hat v) = 
\partial \hat u\tr T \hat v + \hat u\tr \partial T \hat v + \hat u\tr T \partial \hat v = \\
\hat u\tr \partial T \hat v + \partial \hat u\tr \lambda \hat v  + \lambda \hat u\tr \partial \hat v = \\
\hat u\tr \partial T \hat v + \lambda(\partial \hat u\tr \hat v  + \hat u\tr \partial \hat v) = \\
\hat u\tr \partial T \hat v + \lambda \partial(\hat u\tr \hat v).
\end{multline}

Because $\hat u\tr \hat v = 1$,
we have $\partial (\hat u\tr \hat v) = 0$ and
$\partial \lambda = \hat u\tr \partial T \hat v$.
\end{proof}

% Appendix: Fisher's reproductive value
\subsection*{Appendix B}\label{sec:AppB}
\begin{remark}[Fisher's reproductive value]
Let $M$ be an irreducible column-stochastic matrix and $D$ be a positive diagonal matrix. 
The elements of the left \emph{Perron} eigenvector $\hat u$ of the matrix $MD$ can be regarded as \emph{Fisher's reproductive values}~\citep[pg.~27]{Fisher1930}
\end{remark}

\emph{Fisher's reproductive values} can be understood as follows~\citep[ch.~10]{Grafen2006,Otto2007}.
Consider the dynamics not of frequencies but of absolute population sizes such that the vector of the number of individuals of each type at time $t$ is $n(t)$ and the corresponding frequencies are $f_k(t) = n_k(t) / \sum_i{n_i(t)}$.
The dynamics are
\begin{equation}
n(t) = (MD)^t n(0).
\end{equation}

Let $n(k, t)$ be the vector when the initial population is a single individual of type $k$.
The dynamics are
\begin{equation}
n(k,t) = (MD)^t e_k,
\end{equation}
where $e_k$ is a vector with 1 at position $k$ and 0 elsewhere.

The total population size at time $t$ starting with type $k$ is then
\begin{equation}
N(k,t) = \sum_i{n_i(k,t)} = e\tr (MD)^t e_k.
\end{equation}

Now we can compare the sizes of populations based on what type they started
with:
\begin{equation}
\frac{N(j,t)}{N(k,t)} = 
\frac{e\tr (MD)^t e_j}{e\tr (MD)^t e_k}.
\end{equation}

Now write $MD$ in its Jordan canonical form 
\begin{equation}
A = V \Lambda U\tr,
\end{equation}
where $V$ is the matrix of right (column) eigenvectors of $MD$,
$U\tr$ is the transposed matrix of left (row) eigenvectors of $MD$,
where we can take $V U\tr = U\tr V = I$, 
and $\Lambda$ is the diagonal matrix of eigenvalues of $A$ 
(for a non-generic set of matrices $M$, the geometric and algebraic multiplicities of the eigenvalues of $MD$ differ, and $\Lambda$ will not be a diagonal matrix, a case we can ignore).

Hence,
\begin{multline}
N(k, t) = 
e\tr (MD)^t e_k =
e\tr  (V \Lambda U\tr)^t e_k = \\
e\tr V \Lambda^t U\tr e_k.
\end{multline}

For the ratio, we can divide $\Lambda$ by $\lambda_1 = \rho(MD)$,
the spectral radius of $MD$:
\begin{multline}
\frac{N(j,t)}{N(k,t)} =
\frac{ e\tr V \Lambda^t U\tr e_j}{ e\tr V \Lambda^t U\tr e_k} = \\
\frac{ e\tr V \diag{1, \Big(\frac{\lambda_2}{\lambda_1}\Big)^t, \cdots} U\tr e_j} { e\tr V \diag{1,\Big(\frac{\lambda_2}{\lambda_1}\Big)^t, \cdots} U\tr e_k}.
\end{multline}
Now take the limit $t \goesto \infty$.
By assumption, $MD$ is irreducible, so $\lambda_i < \lambda_1$ for all $i > 1$.
Therefore,
\begin{equation}
\lim_{t \goesto \infty} \Big( \frac{\lambda_k}{\lambda_1} \Big)^t = 0
\end{equation}
for all $k > 1$, and
\begin{equation}
\lim_{t \goesto \infty}\frac{N(j,t)}{N(k,t)} =
\frac{e\tr V (e_1 e_1\tr) U\tr e_j} { e\tr V (e_1 e_1\tr) U\tr e_k} = 
\frac{ e\tr \hat v \hat u_j} { e\tr \hat v  \hat u_k} =
\frac{\hat u_j} {\hat u_k}.
\end{equation}

The vector $\hat u$ is the left \emph{Perron} eigenvector of $MD$, and $\hat u_k$ is $k$-th element of $\hat u$.
This is why the value $\hat u_k$ can be interpreted as the \emph{reproductive value} of type $k$: it is a weighting for the size of the population generated by a single individual of type $k$.

If we begin with a population at the equilibrium distribution $\hat v$,
and ask what fraction of long-term descendants descended from type $k$ at that time,
we weight the equilibrium frequency $\hat v_k$ by the reproductive value $\hat u_k$, to get $\hat u_k \hat v_k$.
$\{\hat u_k \hat v_k\}_{k}$ is a probability distribution, since
\begin{equation}
\sum_k {\hat u_k \hat v_k} = \hat u\tr \hat v = 1.
\end{equation}
\citet{Hermisson2002} called this distribution the \emph{ancestor} or \emph{ancestral distribution}.

% Appendix: inner stability
\subsection*{Appendix C}\label{sec:AppC}

\begin{lemma}[Stability of the equilibrium $\hat v$]
If $C$ and $D$ are positive diagonal matrices, and $M$ is primitive, 
then the equilibrium $\hat v$ of the system in Eq.~\ref{eq:model_equilibrium} is globally stable.
\end{lemma}

\begin{proof}
Let $A=(I-C+MC)D$ and denote the leading eigenvalue of $A$ as $\hat{\bar w}$.

According to the \emph{Perron-Frobenius theorem}, 
\begin{equation}
\bigg(\frac{A}{\hat{\bar w}}\bigg)^t \goesto \frac{\hat v \hat u\tr}{\hat u\tr \hat v} = \hat v \hat u\tr,
\end{equation}
where $\hat u$ and $\hat v$ are the left and right Perron eigenvectors of $A$ such that 
\begin{equation}
\sum_{k=1}^{n}{\hat v_k}=1, \quad
\hat u\tr \hat v=\sum_{k=1}^{n}{\hat u_k \hat v_k}=1,
\label{eq:sum_uv}
\end{equation}
and $\hat v \hat u\tr$ is the \emph{Perron projection} into the eigenspace corresponding to $\hat{\bar w}$.

Therefore, for any positive frequency vector $f$
\begin{equation}
\bigg(\frac{A}{\hat{\bar w}}\bigg)^t \cdot f \goesto \hat v \hat u\tr \cdot f = \alpha \hat v,
\label{eq:perron_projection}
\end{equation}
where $\alpha \in \mathbb{R}$.
Using eqs.~\ref{eq:sum_uv} and~\ref{eq:perron_projection}
\begin{equation}
\alpha = \alpha \hat u\tr \hat v = \hat u\tr \alpha \hat v = \hat u\tr \hat v \hat u\tr f = \hat u\tr f = \sum_{k=1}^{n}{\hat u_k f_k} > 0,
\label{eq:alpha0}
\end{equation}
because $f$ and $\hat u$ are positive vectors.

Now, let $\norm{x} = \sum_{k=1}^{n}\abs{x_k}$, so that $x_t \goesto x$ implies
\begin{equation}
\norm{x_t} \goesto \norm{x},
\label{eq:norm_limit}
\end{equation}
as $t \goesto \infty$ and rewrite Eq.~\ref{eq:model} as 
\begin{equation}
f' = \frac{A f}{\norm{A f}}. 
\label{eq:model_norm}
\end{equation}

It is easily seen that $f^t =A^tf/\|A^tf\|$. Hence 
\begin{multline}
\frac{A^t f}{\norm{A^t f}} = 
\frac{\frac{1}{(\hat{\bar w})^t} A^t f}{\frac{1}{(\hat{\bar w})^t} \norm{A^t f}} = 
\frac{\big(\frac{A}{\hat{\bar w}}\big)^t \cdot f }{ \norm{ \big(\frac{A}{\hat{\bar w}}\big)^t \cdot f }} \goesto
\frac{\alpha \hat v}{\norm{\alpha \hat v}} = \hat v,
\end{multline}
since $\norm{\hat v}=1$ (eq.~\ref{eq:sum_uv}) and $\alpha \ne 0$ (eq.~\ref{eq:alpha0}).
This completes the proof.
\end{proof}

% Acknowledgements
{\small
\section*{Acknowledgements}

This work was supported in part by 
the Department of Information and Computer Sciences at the University of Hawai`i at M{\=a}noa, 
the Konrad Lorenz Institute for Evolution and Cognition Research, 
the Mathematical Biosciences Institute through National Science Foundation Award \#DMS 0931642, 
the Stanford Center for Computational, Evolutionary and Human Genomics, 
and the Morrison Institute for Population and Resources Studies, Stanford University.
}

\bibliographystyle{agsm}
%\bibliography{/Users/yoavram/Documents/library}
\bibliography{ms}

\end{document}  