\documentclass[9pt, a4paper, twocolumn]{extarticle}
\usepackage[margin=1in]{geometry} % See geometry.pdf to learn the layout options.
\geometry{a4paper}
%\usepackage[parfill]{parskip} % Activate to begin paragraphs with an empty line rather than an indent
\usepackage{graphicx} % Use pdf, png, jpg, or eps§ with pdflatex; use eps in DVI mode
% TeX will automatically convert eps --> pdf in pdflatex		

\usepackage{amssymb,amsmath,amsthm}
\usepackage{longtable}
\usepackage[hyphens]{url}
\PassOptionsToPackage{hyphens}{url} % url is loaded by hyperref
\usepackage[unicode=true]{hyperref}
%\usepackage{nameref} % included in hyperref
\usepackage[auth-sc]{authblk}

% messes up nameref
%\usepackage{titlesec}
%\titleformat{\subsection}[runin]
%  {\bf}{\thesection}{1em}{}
  
%SetFonts
% newtxtext+newtxmath
\usepackage{newtxtext} %loads helv for ss, txtt for tt
\usepackage{amsmath}
\usepackage[bigdelims]{newtxmath}
\usepackage[T1]{fontenc}
\usepackage{textcomp}
%SetFonts

% Yoav & Lee commands
\newcommand*{\tr}{^\intercal}
\let\vec\mathbf
\newcommand{\matrx}[1]{{\left[ \stackrel{}{#1}\right]}}
\newcommand{\diag}[1]{\mbox{diag}\matrx{#1}}
\newcommand{\goesto}{\rightarrow}
\newcommand{\dspfrac}[2]{\frac{\displaystyle #1}{\displaystyle #2} }
\newtheorem*{theorem}{Theorem}
\newtheorem*{corollary}{Corollary}
\newtheorem{lemma}{Lemma}
\newtheorem*{remark}{Remark}
\newtheorem*{mfp}{Theorem: Mean Fitness Principle}
\newtheorem*{rvp}{Corollary: Reproductive Value Principle}
\newtheorem*{frv}{Fisher's reproductive values}
\renewcommand\qedsymbol{} % no square at end of proof

% NatBib
\usepackage[round,colon,authoryear]{natbib}

% Title page
\title{Variation generation and mean fitness increase: \\ Necessity is the mother of genetic invention}

\author[a]{Yoav Ram}
\author[b]{Lee Altenberg}
\author[a]{Marcus W. Feldman}
\affil[a]{Department of Biology, Stanford University, Stanford, CA}
\affil[b]{Information and Computer Sciences, University of Hawai`i at M{\=a}noa, Honolulu, HI}

\date{\today}

% Document
\begin{document}
\maketitle

% Abstract
\begin{abstract}
Generation of variation may be detrimental in well-adapted populations evolving under constant selection.
In a constant environment, genetic modifiers that reduce the rate at which variation is generated by processes such as mutation and migration, succeed.
However, departures from this \emph{reduction principle} have been demonstrated.
Here we analyze a general model of evolution under constant selection where the rate at which variation is generated can depend on the fitness of the individual.
We find a general and simple result:
if individuals with below-average fitness increase the rate at which variation is generated, then the population mean fitness increases.
This principle applies to phenomena such as stress-induced mutagenesis, condition-dependent dispersal and outcrossing,
and even to the proverb: \emph{"Necessity is the mother of genetic invention"}.
\end{abstract}

% Introduction
\section*{Introduction}

According to the \emph{reduction principle}, in populations that evolve near an equilibrium between natural selection and a process that generates variation (i.e. mutation, migration, or recombination), selection favors neutral modifiers that decrease the rate at which variation is generated. 

The \emph{reduction principle} was demonstrated for modifiers of the recombination~\citep{Feldman1972}, mutation, and migration~\citep{Liberman1986a}, and dispersal rates~\citep{Altenberg1987}.
These separate principles were recently unified by~\citet{Altenberg2017}

The assumptions and conditions for the validity of the \emph{unified reduction principle} by neutral genetic modifiers are~\citep{Altenberg2017}:
(i) effectively infinite population size, (ii) constant-viability selection, (iii) a population at an equilibrium, and (iv) \emph{linear variation} -- the equal scaling of transition probabilities by the modifier.
A departure from the latter assumption occurs if two variation producing processes interact~\citep{Feldman1980,Altenberg2012}.
Departures from the \emph{reduction principle} have also been demonstrated when conditions (i)-(iii) are not met, see for example~\citet{Holsinger1986} and references therein.

Another departure from the \emph{linear variation} assumption of the \emph{reduction principle} for mutation rates involves \emph{stress-induced mutagenesis}, a mechanism in which the mutation rate increases in individuals with poor fitness.
Empirical evidence suggests \emph{stress-induced mutagenesis} is common in bacteria and yeast, and may be prevalent in plants, flies, and human cancer cells~\citep{Rosenberg2012,Fitzgerald2017b}.
\citet{Ram2012} demonstrated that modifiers that increase the mutation rate of individuals with below-average fitness actually \textbf{increase} the population mean fitness, rather than decrease it.
Their analysis assumed an infinite population size and constant-viability selection, where the only departure from the \emph{reduction principle} assumptions is due to the differential effect the modifier has on different genotypes.
A similar result has been demonstrated for fitness-associated recombination~\citep{Hadany2003a} and for condition-dependent sexual reproduction~\citep{Hadany2007a}, and evidence suggests that both mechanisms are common in nature~\citep{Ram2016}.

\citet{Ram2012} stated that their result represents a departure from the \emph{reduction principle}, but did not explain this departure.
Their analysis was specific to a model that classified individuals by the number of mutant genes in their genotype, similar to models studied by~\citet{Kimura1966a} and~\citet{Haigh1978}.

Here, we present a general evolutionary model where the rate at which variation is generated depends on the type of individual, and specifically on the its fitness.
Our result shows that if below-average individuals produce more variation than above-average individuals, then the population mean fitness increases.

% Model
\section*{The Models}

\paragraph*{General model.}\label{sec:general_model}
Consider a large haploid population with an arbitrary (possibly infinite) set of types $A_1, A_2, \ldots$.
The change in the frequencies of these types is described by the transformation $f \to f'$: 
\begin{equation}\label{eq:model}
\bar{w} f' = (I-C+MC)D f,
\end{equation}
where $f_k$ is the frequency of type $A_k$ ($f=(f_1, f_2, \ldots)$ is a stochastic vector: $\sum_k{f_k} = 1$);
$C_k$ is the probability that an individual of type $A_k$ will transition to another type ($C$ is a diagonal non-negative matrix with diagonal elements $C_k$);
$M_{i,j}$ is the transition probability from type $A_j$ to type $A_i$ given that an individual of type $A_j$ is transitioning ($M$ is a column-stochastic matrix: $\sum_j {M_{i,j}} = 1$ for all \emph{j}); 
$D_k$ is the relative fitness of type $A_k$ ($D$ is a diagonal matrix with diagonal elements $D_k$, where we can assume $0 < D_k \le 1$, i.e., fitness values are relative to the fitness of the fittest type); 
$I$ is the identity matrix;
and $\bar{w}$ is the normalizing factor such that $\sum_k{f'_k}=1$ and is equal to the population mean fitness (see~\nameref{sec:AppA}).

The types $A_k$ can represent a single or multiple genetic loci or non-genetic traits.
Importantly, type transmission is vertical and uni-parental (the type is transmitted from a single parent to the offspring), independent of the frequencies of the other types, and is fully described by the matrix $M$. Specifically, transmission infidelity causes type transitions.
Therefore, this model precludes processes such as recombination, social learning, sexual outcrossing, and any horizontal or oblique transmission, as these processes are frequency-dependent~\cite[pg.~54]{Cavalli-Sforza1981}.

Transition between types is determined by a combination of two effects:
(i) the probability to transition \emph{out} of type $A_k$ is determined by $C_k$;
(ii) given a transition out of type $A_k$, the distribution of the destination types $A_j$ is given by $M_{j,k}$ (note the index order).
We focus on a modifier that has no direct effect on fitness and whose sole function is to determine the rate of transition out of the different types ($C_k$).
Importantly, this modifier can determine different rates for different types; that is, $C_j \ne C_i$ for some $i,j$. The case $C_j = C_j$ for all $i,j$ is covered by the \emph{reduction principle}, see~\citet{Altenberg2017}.

In the following section we present three particular examples of this model for the modification of mutation, migration, and learning.

\paragraph*{Mutation modification model 1.}\label{sec:mut_model_1}

Here we consider a large population of haploids and a trait determined by a major locus with $n$ possible alleles $A_1, A_2, ..., A_n$.
Linked to to this locus is a modifier locus that determines the mutation rates $\mu_k$ of individuals with allele $A_k$.
Specifically, with probability $1-\mu_k$, the allele $A_k$ does not mutate, and with probability $\frac{\mu_k}{n-1}$, the allele $A_k$ mutates to $A_j$ for any $j \ne k$.
This is an extension of a model studied by~\citet{Altenberg2017}, that allows for the mutation rate of an individual, $\mu_k$, to depend on the identity of the allele $A_k$.

Let the frequency of $A_k$ in the present generation be $f_k$ with $f_k \ge 0$ and $\sum_k{f_k}=1$.
Then after selection and mutation, $f'_k$ in the next generation is given by
\begin{equation}
f_k' = (1 - \mu_k) \frac{w_k}{\bar{w}} f_k + \frac{1}{n-1} \sum_{k \ne j}{\mu_j \frac{w_j}{\bar{w}} f_j},
\label{eq:mutation_model_1}
\end{equation}
for all $k=1,2,\ldots,n$, where $\bar{w} = \sum_{k=1}^{n}{w_k f_k}$ is the population mean fitness.

This model can be rephrased in terms of the general model: 
Eq.~\ref{eq:mutation_model_1} is equivalent to Eq.~\ref{eq:model} if we define
\begin{equation} 
C_k = \mu_k, \quad
D_k = w_k, \quad
M_{i,j} = \begin{cases}
\frac{1}{n-1}, & i \ne j \\
0, & i=j
\end{cases}.
\end{equation}

\paragraph*{Mutation modification model 2.}\label{sec:mut_model_2}

Here, we consider a large population of haploids in which individuals with genotype $A_k$ are characterized by the number $k$ of deleterious or mutant alleles in their genotype.
A modifier locus then determines the probability $m_k$ that a mutation occurs in individuals with $k$ mutant alleles.
In this case, we assume a low mutation rate $m_k$ so that two mutations are unlikely to occur in the same individual in one generation.
When a mutation occurs it is \emph{deleterious} with probability $\delta$, generating a mutant allele and converting the individual from $A_k$ to $A_{k+1}$, or it is beneficial with probability $\beta$, converting the individual from $A_k$ to $A_{k-1}$.
Note that such beneficial mutations can be either compensatory or back-mutations, and that mutations are neutral with probability $1-\delta-\beta$.
This model has been analyzed by~\citet{Ram2012}.

Let the frequency of $A_k$ in the present generation be $f_k$ such that $f_k \ge 0$ and $\sum_k{f_k}=1$.
Then after selection and mutation $\vec{f'}$ in the next generation is given by
\begin{equation}
f_0' = (1 - \delta m_0) \frac{w_0}{\bar{w}} f_0 + \beta m_{1} \frac{w_{1}}{\bar{w}} f_{1},
\end{equation}
and
\begin{multline}
f_k' = (1 - m_k(\delta+\beta)) \frac{w_k}{\bar{w}} f_k + \\
\delta m_{k-1} \frac{w_{k-1}}{\bar{w}} f_{k-1} + \beta m_{k+1} \frac{w_{k+1}}{\bar{w}} f_{k+1},
\label{eq:mutation_model_2}
\end{multline}
for all $k=1,2,\ldots$, where $w_k$ is the fitness of an individual with $k$ mutant alleles and $\bar{w} = \sum_{k \ge 0}{w_k f_k}$ is the population mean fitness.

In terms of the general model, we can write
\begin{equation}
C_k = m_k, \quad
D_k = w_k,
%M_{i,j} = \begin{cases}
%\beta, & i = j-1 \\
%1-\beta-\delta, & i = j \\
%\delta, & i = j+1 \\
%1-\delta, & i=j=0 \\
%0, & \text{otherwise}
%\end{cases}
\end{equation}
and
\begin{equation}
M_{i,j} = \begin{bmatrix}
1-\delta & \beta & 0 &  \cdots \\
\delta & 1-\delta-\beta & \beta &  \ddots \\
0 & \delta & 1-\delta-\beta & \ddots \\
\vdots & \ddots & \ddots & \ddots 
\end{bmatrix},
\end{equation}
in which case Eq.~\ref{eq:mutation_model_2} is a specific case of Eq.~\ref{eq:model}.


\paragraph*{Migration modification model.}\label{sec:mig_model_1}

Here we assume a large population of haploids that occupies two demes and a modifier locus that determine the migration rates between the demes, $m_k$. 
Let the fitness of genotypes $A_k$ be $w_k$ in deme 1 and $q_k$ in deme 2 for $k=1,2,\ldots,n$.
Let the frequencies of $A_k$ be $x_k$ in deme 1 and $y_k$ in deme 2 for $k=1,2,\ldots,n$ with $x_k \ge 0$, $y_k \ge 0$, $\sum_{k=1}^{n}{x_k} = 1$, and $\sum_{k=1}^{n}{y_k} = 1$.

Then following selection and migration, after one generation the new frequencies $x_k'$ and $y_k'$ are given by
\begin{align}
x_k' = (1-m_k) \frac{w_k}{\bar{w}} x_k + m_k \frac{q_k}{\bar{q}} y_k \\
y_k' = m_k \frac{w_k}{\bar{w}} x_k + (1 - m_k) \frac{q_k}{\bar{q}} y_k
\end{align}
for $k=1,2,\ldots,n$, where $\bar{w} = \sum_{k \ge 0}{w_k x_k}$ and $\bar{q} = \sum_{k \ge 0}{q_k y_k}$ are the population mean fitnesses in deme 1 and 2, respectively.
This is an extension of a model presented by~\citet{Altenberg2017}, allowing for the migration rate of an individual, $m_k$, to depend on the identity of the allele $A_k$.

To rephrase this model in terms of the general model, we define
\begin{equation}
\begin{aligned}
f = (x_1, \ldots, x_n, y_1, \ldots, y_n)\tr, \\
C = diag[m_1, \ldots, m_n, m_1, \ldots, m_n], \\
D = diag[w_1, \ldots, w_n, q_1, \ldots, q_n], \\
M = \begin{bmatrix}
0 & I_n \\
I_n & 0
\end{bmatrix}
\end{aligned}
\end{equation}
where $I_n$ is the n-by-n identity matrix.

% Modifier model
\paragraph*{Modifier model.}
We now consider the case of two modifier alleles, $m$ and $M$, in complete linkage to the types $A_k$.
These modifiers potentially induce different transition probabilities out of types $A_k$, with transition matrices $C=diag[C_1, \ldots, C_n]$ and $\tilde{C}=diag[\tilde{C}_1, \ldots, \tilde{C}_n]$.
The frequencies of type $A_k$ linked to modifier $m$ or $M$ is $f_k$ and $g_k$, respectively, such that $\sum_k{f_k+g_k}=1$;
$\bar{w}$ now ensures that $\sum_{k}{f_k'+g_k'}=1$;
the rest of the model parameters are the same as in Eq.~\ref{eq:model};
and the frequencies in the next generation, $f'$ and $g'$, are given by
\begin{equation}
\begin{cases}
\bar{w} f' &= (I-C+MC)D f \\
\bar{w} g' &= (I-\tilde{C}+M\tilde{C})D g
\end{cases}.
\label{eq:modifier_model}
\end{equation}
Note that Eq.~\ref{eq:model} is a special case for $g_k=0$ for all $k$.
Here, $\bar{w}=\sum_{k}{(f_k+g_k)D_k}$ is the mean fitness of the entire population (see~\nameref{sec:AppA}). 

% Results
\section*{Results}

% Mean fitness principle
Write the equilibrium of the frequency vector $f$ in Eq.~\ref{eq:model} as $v$ and the stable population mean fitness as $\bar{w}^*$, then
\begin{equation}\label{eq:model_equilibrium}
\bar{w}^* v = (I-C+MC)D v
\end{equation}

The following theorem presents a \emph{mean fitness principle} for the
sensitivity of the stable population mean fitness $\bar{w}^*$ to changes
in $C_k$ the probability of transition from $A_k$.

\begin{mfp}
Let $\bar{w}^*$ be the leading eigenvalue of $(I-C+MC)D$, $u$ and $v$ be the corresponding left and right eigenvectors, and $[M]_k$ the $k$-th column of $M$. Then
\begin{equation}\label{eq:theorem}
\frac{\partial \bar{w}^*}{\partial C_k} = 
\frac{u_k v_k}{C_k} (\bar{w}^* - D_k),
\end{equation}
or in simpler terms,
\begin{equation}\label{eq:sign_theorem}
sign\frac{\partial \bar{w}^*}{\partial C_k} = 
sign(\bar{w}^* - D_k).
\end{equation}

Therefore increased transition from type $k$ will increase the stable population mean fitness if and only if the fitness of type $k$ is below the stable population mean fitness.
\end{mfp}

\begin{proof}
By the \emph{Perron-Frobenius theorem}~\cite[Appendix~A]{Otto2007},
$\bar{w}^*$ is the leading real positive eigenvalue of $(I-C+MC)D$,
and the corresponding left and right eigenvectors $u$ and $v$ are both non-negative and uniquely defined by $\sum_k{v_k} = 1$ and $\sum_k{u_k v_k} = 1$.

Using \emph{Caswell's formula} (Eq.~\ref{eq:Caswells_formula} in~\nameref{sec:AppB}),
\begin{equation}
\frac{\partial \bar{w}^*}{\partial C_k} = 
u\tr \frac{\partial (I-C+MC)D}{\partial C_k} v.
\end{equation}

Let $e_k$ and $e\tr_k$ be the column and row vectors with 1 at position $k$ and 0 elsewhere, $Z_k = e_k e\tr_k$ the matrix with 1 at position $(k,k)$ and 0 elsewhere, and $[M]_k$ the $k$-th column of $M$.

Then,
\begin{equation}\label{eq:theorem_intermediate}
\begin{aligned}
u\tr \frac{\partial (I-C+MC)D}{\partial C_k} v = 
u\tr (0 - Z_k + M Z_k)D v = \\
-v_k u_k D_k + v_k D_k u\tr[M]_k = \\
v_k D_k (u\tr[M]_k - u_k). 
\end{aligned}
\end{equation}

The corresponding equation to Eq.~\ref{eq:model_equilibrium} for the left
eigenvector $u$ is 
\begin{equation}
u\tr \bar{w}^* = u\tr (I - C + MC) D,
\end{equation}
which gives us a relation between $\bar{w}^*$ and the $k$
element of $u$:
\begin{equation}
u_k \bar{w}^* = (1-C_k) D_k u_k + C_k D_k u\tr [M]_k.
\end{equation}
Multiplying both sides by $v_k$ (which is positive if $D_k>0$) and rearranging, we get 
\begin{equation}
\frac{u_k v_k}{C_k} (\bar{w}^* - D_k) = v_k D_k(u\tr [M]_k - u_k),
\end{equation}
which when substituted into Eq.~\ref{eq:theorem_intermediate} yields:
\begin{equation}
\frac{\partial \bar{w}^*}{\partial C_k} = 
\frac{u_k v_k}{C_k} (\bar{w}^* - D_k).
\end{equation}
Finally, note that $u_k, v_k, C_k \ge 0$ and therefore
\begin{equation}
sign \frac{\partial \bar{w}^*}{\partial C_k} = 
sign (\bar{w}^* - D_k),
\end{equation}
which completes the proof.
\end{proof}

The above theorem provides a condition for the effect of a modifier in control of $C_k$, the probability for transition from $A_k$, on the stable population mean fitness.
Specifically, if $A_k$ individuals have below-average fitness, then a modifier that increases $C_k$ will increase the population mean fitness.

% Reproductive value principle

An interesting intermediate result in Eq.~\ref{eq:theorem_intermediate} is

\begin{equation}
\begin{aligned}
\frac{\partial \bar{w}^*}{\partial C_k} = 
v_k D_k (u\tr[M]_k - u_k) = \\
D_k (\sum_i{u_i M_{i,k} v_k} - u_k v_k).
\end{aligned}
\end{equation}
Here, $u_k$ can be regarded as the \emph{reproductive value}~\citep[pg.~27]{Fisher1930} of type $k$ (see~\nameref{sec:AppC}), which gives the relative contribution of type $k$ to the long-term population.
Consequently, $u_k v_k$ is the \emph{ancestor frequency} \citep{Hermisson2002} of type $k$, namely the fraction of the equilibrium population descended from type $k$.
The sum $\sum_i{u_i M_{i,k} v_k}$ can be similarly interpreted as the fraction of the equilibrium population descended from individuals that transitioned from type $k$ to another type (via the $k$ column of the transition matrix $M$).

Given $0 < D_k \le 1$, from Eq.~\ref{eq:theorem_intermediate} we can get the following corollary.

\begin{rvp}
In the notation of the theorem,
\begin{equation}
sign \frac{\partial \bar{w}^*}{\partial C_k} = 
sign (u\tr [M]_k - u_k),
\end{equation}
and therefore increased transition out of type $k$ will increase the stable population mean fitness if and only if the fraction of the population descended from type $k$ is expected to increase due to a transition to another type. 
\end{rvp}

% Threshold rule

We turn our attention to a modifier that concurrently controls the transition out of a subset of types, $K$.
For example, \citet[Appendix~B]{Ram2012} considered individuals that are grouped by the number of their accumulated deleterious mutations, $k$, and a modifier that regulates the mutation rates in individuals with at least $\pi$ deleterious mutations.
Similarly, if the transition rates are controlled by a threshold rule 
\begin{equation}
C_k = \begin{cases}
\tau, & k \in K \\
1, & k \not\in K
\end{cases},
\end{equation}
where $\tau > 0$, then the sensitivity of the stable population mean fitness to changes in the
rate of transition from types $K$ can be written 
\begin{equation}
\begin{aligned}
\frac{\partial \bar{w}^*}{\partial \tau} = 
u\tr \Big( \sum_{k \in K}{\frac{\partial (I - C + MC)D}{\partial \tau}} \Big) v = \\
\frac{1}{\tau} \sum_{k \in K}{u_k v_k (\bar{w}^* - D_k)},
\end{aligned}
\end{equation}
which can be simplified to
\begin{multline}\label{eq:sign_theorem_threshold}
sign \frac{\partial \bar{w}^*}{\partial \tau} = 
sign \sum_{k \in K}{u_k v_k (\bar{w}^* - D_k)} = \\ 
sign \Big(\bar{w}^* - \frac{\sum_{k \in K}{u_k v_k D_k}}{\sum_{k \in K}{u_k v_k}}\Big), 
\end{multline}
a generalization of Eq.~\ref{eq:sign_theorem}.
Therefore, increased transition from types $K$ will increase the stable population mean fitness if and only if the average fitness of individuals descended from types $K$ is below the stable population mean fitness.

If such a modifier has the same effect on all individuals (i.e., $C_k = \alpha$ for all $k$) then Eq.~\ref{eq:model_equilibrium} $\bar{w}^* v = ((1-\alpha)I + \alpha M)Dv$,
and the \emph{unified reduction principle} is in effect~\cite[eqs. 65, 72]{Altenberg2017}.
Therefore, we can expect modifiers that reduce $\alpha$ to be favored by natural selection.
Indeed, setting $C = \alpha I$ in Eq.~\ref{eq:model_equilibrium} and proceeding as in the above proof, we find
$$
\frac{\partial \bar{w}^*}{\partial \alpha} = 
\bar{w}^* - 1 \le 0,
$$
because $\bar{w}^* = \sum_i{v_i D_i}$ (see~\nameref{sec:AppA}), and $D_i \le 1$ for all $i$.

% Evolutionary stability

We determined a condition for a modifier that controls that transition rate $C_k$ from type $A_k$ to increase the stable population mean fitness.
But could such a modifier increase in frequency when initially rare in the population?
To answer this question we will analyze the stability of modifier $m$ with transition rates $C_k$ to the invasion of modifier $M$ with rates $\tilde{C}_k$, as described in Eq.~\ref{eq:modifier_model}.

The equilibrium of Eq.~\ref{eq:modifier_model} when $M$ is missing from the population ($g_k=0$ for all $k$) is (modified from \ref{eq:model_equilibrium})
\begin{equation}\label{eq:modifier_model_equilibrium}
\bar{w} \cdot \begin{pmatrix}v \\ 0\end{pmatrix} = 
\Big(\vec{I} - \vec{C} + \vec{M} \vec{C}\Big)\vec{D} \begin{pmatrix}v \\ 0\end{pmatrix},
\end{equation}
where $\vec{I}$ is a $2n$-by-$2n$ identity matrix, and
\begin{equation}
%\begin{aligned}
\vec{C} =  \begin{pmatrix}C & 0 \\ 0 & \tilde{C} \end{pmatrix}, \quad
\vec{M} =  \begin{pmatrix}M & 0 \\ 0 & M \end{pmatrix}, \quad
\vec{D} =  \begin{pmatrix}D & 0 \\ 0 & D \end{pmatrix}.
%\end{aligned}
\end{equation}

The stability of modifier $m$ to invasion by modifier $M$ is determined by the leading eigenvalue $\lambda$ of the external stability matrix $L_{ex}$ of modifier $M$, which is in turn determined by the Jacobian of the system in Eq.~\ref{eq:modifier_model_equilibrium}
\begin{equation}
J = \begin{pmatrix}L_{in} & 0 \\ 0 & L_{ex} \end{pmatrix}.
\end{equation}
Note the zero block matrices due to complete linkage between the modifier and the types $A_k$.

$L_{ex}$ can be written 
\begin{equation}\label{eq:Lex}
L_{ex} = (I - \tilde{C} + M \tilde{C}) D /\bar{w}^*,
\end{equation}
where $\bar{w}^*$ is the stable population mean fitness without modifier $M$ (see~\nameref{sec:AppA}). 

\begin{theorem}[Evolution of increased genetic variation]
The sign of the derivative of the leading eigenvalue $\lambda$ of the external stability matrix of $L_{ex}$ of modifier $M$, when evaluated at $\tilde{C}_k=C_k$, is
\begin{equation}
sign \frac{\partial \lambda}{\partial \tilde{C}_k}\bigg\rvert_{\tilde{C}_k = C_k} = 
sign(\bar{w}^* - D_k).
\end{equation}
In addition, the leading eigenvalue is $\lambda=1$ when transition rates are equal $\tilde{C}_k=C_k$ for all $k$.

Therefore, an invading modifier $M$ with transition rates slightly different from the resident modifier $m$ can successful invade the population ($\lambda>1$ if $M$ increases the probability of transition from types with below-average fitness.
\end{theorem}

\begin{proof}
First, consider the case the $\tilde{C_k} = C_k$ for all $k$.
Substituting $\tilde{C}=C$ in Eq.~\ref{eq:Lex} and multiplying both sides by $\bar{w}^*$,
\begin{equation}
\bar{w}^* L_{ex} = (I-C+MC) D,
\end{equation}
and since $\bar{w}^*$ is the leading eigenvalue of the RHS (\nameref{sec:AppA}),
then the leading eigenvalue of $L_{ex}$ is $\lambda=1$.

Now, applying Eq.~\ref{eq:sign_theorem} on Eq.~\ref{eq:Lex}, the sign of the derivative of $\lambda$ with respect to $\tilde{C_k}$ is
\begin{equation}
sign \frac{\partial \lambda}{\partial \tilde{C}_k} =
sign\Big(\lambda - \frac{D_k}{\bar{w}^*}\Big),
\end{equation}
and evaluating this expression at $\lambda=1$ (the case for $\tilde{C}_k = C_k$) gives us
\begin{equation}
sign \frac{\partial \lambda}{\partial \tilde{C}_k}\bigg\rvert_{\lambda=1} = 
sign(\bar{w}^* - D_k),
\end{equation}
since $\bar{w}^*>0$.
This completes the proof.
\end{proof}

% Discussion
\section*{Discussion}

We have shown that under constant-viability selection and in an effectively infinite haploid population at mutation-selection or migration-selection equilibrium, the stable population mean fitness increases if below-fitness individuals increase the rate at which variation is generated.
We have given several examples of variation-generating processes for which this principle applies -- mainly mutation and migration -- but our model may apply to other processes as well.
For example, the \emph{reduction principle} applies to ecological models of dispersal, and~\citet{Gueijman2013} have demonstrated that even in homogeneous environments, fitness-associated dispersal increases the population mean fitness and is favored by selection over uniform dispersal.
Similarly, if the transmission fidelity of culturally-transmitted traits depends on the type of the type or fitness of the transmitting individual, we expect that our \emph{mean fitness principle} will hold.

Eq.~\ref{eq:theorem} is a generalization of a result of~\citet[Eq.~4]{Ram2012}.
\citeauthor{Ram2012} modeled the accumulation of mutant alleles in a population evolving at the mutation-selection balance (see \nameref{sec:mut_model_2}).
Using Eq.~\ref{eq:Caswells_formula} in \nameref{sec:AppB} and a recursion on the ratios of the reproductive values \cite[see][eqs.~A5-6]{Ram2012}, they concluded that at the mutation-selection balance, if individuals with below-average fitness ($D_k < \bar{w}^*$) increase their mutation rate, then the population mean fitness will increase -- a result generalized by our \emph{mean fitness principle} in Eqs.~\ref{eq:sign_theorem} and~\ref{eq:sign_theorem_threshold}.

Our analysis focuses on populations at equilibrium.
Nevertheless, it has been demonstrated that a modifier which increases the mutation rate of maladapted individuals can be favored by selection~\citep{Ram2012} and reduce the adaptation rate~\citep{Ram2014}.
Similar results have been demonstrated for a modifier that increases the recombination rate in maladapted individuals~\citep{Hadany2003a,Hadany2003b}.

% Conclusions
\subsection*{Conclusions}

General examples of departures from the \emph{reduction principle} are rare~\citep{Altenberg2017}.
However, it has been demonstrated that higher mutation, recombination, and migration rates are favored by selection in populations evolving under fluctuating selection; see~\citet{Carja2014} and references therein.
Here we have provided another general example, which suggests that if maladapted or below-average individuals increase the rate at which variation is generated, the population mean fitness will increase.

\section*{Appendices}
% Appendix A: mean fitness
\subsection*{Appendix A}\label{sec:AppA}

\begin{lemma}
The normalizing factor $\bar{w}$ in Eq.~\ref{eq:model} is the population mean fitness
\begin{equation}
\bar{w} = \sum_k{f_k D_k}.
\end{equation}
\end{lemma}

\begin{proof}
By definition, the normalizing factor $\bar{w}$ ensures that $\sum_k{f'_k}=1$.
Denote $e = (1, ..., 1)$ the vector with ones at all positions and rewrite $e\tr f' = 1$.
Therefore, 
\begin{multline}\label{eq:mean_fitness}
\bar{w} = 
\bar{w} e\tr f' = 
e\tr (I - C + M C) D f = \\
(e\tr I - e\tr C + e\tr M C) D f = \\
(e\tr - e\tr C + e\tr C) D f = \\
e\tr D f = 
\diag{D} f = 
\sum_k{f_k D_k},
\end{multline}
using the fact that $e\tr M = e\tr$ and $e\tr I = e\tr$ because $M$ and $I$ are stochastic matrices.
\end{proof}

Note that this lemma also ensures that the stable population mean fitness $\bar{w}^*=\sum_k{v_k D_k}$ is the leading eigenvalue of $(I - C + M C) D$; see Eq.~\ref{eq:model_equilibrium} and the \emph{Perron-Frobenius theorem}~\cite[Appendix~A]{Otto2007}.

% Appendix B: Caswell's formula
\subsection*{Appendix B}\label{sec:AppB}

\citet{Caswell1978} gave a \emph{formula for the sensitivity of the population growth rate to changes in life history parameters}.
In this formula, the \emph{population growth rate} is the leading eigenvalue of the population transformation matrix $T$, the \emph{life history parameters} are elements of $T$, and the \emph{sensitivity} is the derivative of the former with respect to the latter.
This is a useful formula~\citep{Caswell1978,Hermisson2002,Ram2012}, and therefore we reproduce it here.

\begin{lemma}[Caswell's formula]
Let $T$ be a non-negative matrix with leading eigenvalue $\lambda$ and left and right eigenvectors $u$ and $v$ such that $\sum{v_k}=1$ and $u\tr v = \sum{u_k v_k} = 1$.
Then the sensitivity of $\lambda$ to changes in any element $t$ of the matrix $T$ is
\begin{equation}\label{eq:Caswells_formula}
\frac{\partial \lambda}{\partial t} = 
u\tr \frac{\partial T}{\partial t} v
\end{equation}
\end{lemma}

\begin{proof} 
Using the lemma assumptions,
$\lambda = \lambda u\tr v = u\tr \lambda v = u\tr T v$ and differentiating both sides we get $\partial \lambda = \partial (u\tr T v)$.
Using the product rule (once in each direction),

\begin{multline}
\partial (u\tr T v) = 
\partial u\tr T v + u\tr \partial T v + u\tr T \partial v = \\
u\tr \partial T v + \partial u\tr \lambda v  + \lambda u\tr \partial v = \\
u\tr \partial T v + \lambda(\partial u\tr v  + u\tr \partial v) = \\
u\tr \partial T v + \lambda \partial(u\tr v).
\end{multline}

Because $u\tr v = 1$,
we have $\partial (u\tr v) = 0$ and
$\partial \lambda = u\tr \partial T v$.
\end{proof}

% Appendix C: Fisher's reproductive value
\subsection*{Appendix C}\label{sec:AppC}
\begin{frv}
Let $M$ be a stochastic matrix and $D$ be a diagonal matrix with positive diagonal elements. 
The elements of the left \emph{Perron} eigenvector $u$ of the matrix $MD$ can be regarded as \emph{Fisher's reproductive values}~\citep[pg.~27]{Fisher1930}
\end{frv}

\emph{Fisher's reproductive values} can be understood as follows~\citep{Grafen2006}.
Consider the dynamics not of frequencies but of absolute population sizes such that the vector of the number of individuals of each type at time $t$ is $n(t)$ and the corresponding frequencies are $f_k(t) = n_k(t) / \sum_i{n_i(t)}$.
The dynamics are
\begin{equation}
n(t) = (MD)^t n(0).
\end{equation}

Let $n(k, t)$ be the vector when the initial population is a single individual of type $k$.
The dynamics are
\begin{equation}
n(k,t) = (MD)^t e_k,
\end{equation}
where $e_k$ is a vector with 1 at position $k$ and 0 elsewhere.

The total population size at time $t$ starting with type $k$ is then
\begin{equation}
N(k,t) = \sum_i{n_i(k,t)} = e\tr (MD)^t e_k.
\end{equation}

Now we can compare the sizes of populations based on what type they started
with:
\begin{equation}
\frac{N(j,t)}{N(k,t)} = \frac{e\tr (MD)^t e_j}{e\tr (MD)^t e_k}.
\end{equation}

Now write $MD$ in its Jordan canonical form 
\begin{equation}
A = V \Lambda U\tr,
\end{equation}
where $V$ is the matrix of right (column) eigenvectors of $MD$,
$U\tr$ is the transposed matrix of left (row) eigenvectors of $MD$,
where we can take $V U\tr = U\tr V = I$, 
and $\Lambda$ is the diagonal matrix of eigenvalues of $A$ 
(for a non-generic set of matrices $M$, the geometric and algebraic multiplicities of the eigenvalues of $MD$ differ, and $\Lambda$ will not be a diagonal matrix, a case we can ignore).

Hence,
\begin{multline}
N(k, t) = 
e\tr (MD)^t e_k =
e\tr  (V \Lambda U\tr)^t e_k = \\
e\tr V \Lambda^t U\tr e_k.
\end{multline}

For the ratio, we can divide $\Lambda$ by $\lambda_1 = \rho(MD)$,
the spectral radius of $MD$:
\begin{multline}
\frac{N(j,t)}{N(k,t)} =
\frac{ e\tr V \Lambda^t U\tr e_j}{ e\tr V \Lambda^t U\tr e_k} = \\
\frac{ e\tr V \diag{1, \Big(\frac{\lambda_2}{\lambda_1}\Big)^t, \cdots} U\tr e_j} { e\tr V \diag{1,\Big(\frac{\lambda_2}{\lambda_1}\Big)^t, \cdots} U\tr e_k}.
\end{multline}
Now take the limit $t \goesto \infty$.
By assumption, $MD$ is irreducible, so $\lambda_i < \lambda_1$ for all $i > 1$.
Therefore,
\begin{equation}
\lim_{t \goesto \infty} \Big( \frac{\lambda_k}{\lambda_1} \Big)^t = 0
\end{equation}
for all $k > 1$, and
\begin{equation}
\lim_{t \goesto \infty}\frac{N(j,t)}{N(k,t)} =
\frac{e\tr V (e_1 e_1\tr) U\tr e_j} { e\tr V (e_1 e_1\tr) U\tr e_k} = 
\frac{ e\tr v u_j} { e\tr v  u_k} =
\frac{u_j} {u_k}.
\end{equation}

The vector $u$ is the left \emph{Perron} eigenvector of $MD$, and $u_k$ is $k$-th element of $u$.
This is why the value $u_k$ can be interpreted as the \emph{reproductive value} of type $i$: it is a weighting for the size of the population generated by a single individual of type $k$.

If we begin with a population at the equilibrium distribution $v$,
and ask what fraction of long-term descendants descended from type $k$ at that time,
we weight the equilibrium frequency $v_k$ by the reproductive value $u_k$, to get $u_k v_k$.
$\{u_k v_k\}_{k}$ is a probability distribution, since
\begin{equation}
\sum_k {u_k v_k} = u\tr v = 1.
\end{equation}
\citet{Hermisson2002} called this distribution the \emph{ancestor} or \emph{ancestral distribution}.

% Acknowledgements
{\small
\section*{Acknowledgements}

This research was supported in part by the Stanford Center for Computational, Evolutionary and Human Genomics, and by the Morrison Institute for Population and Resources Studies, Stanford University.
}

\bibliographystyle{agsm}
%\bibliography{/Users/yoavram/Documents/library}
\bibliography{ms}

\end{document}  